\documentclass{article}
\usepackage[utf8]{inputenc}
\usepackage{tikz}
\usepackage[blue]{tcolorbox}
\usepackage[utf8]{vietnam}
\documentclass{article}
\usepackage{tcolorbox}
\usepackage{listings}
\usepackage{tikz}

\usetikzlibrary{shapes}

\lstset{
  language=C++,
  basicstyle=\small\ttfamily,
  keywordstyle=\color{blue},
  commentstyle=\color{green!50!black},
  stringstyle=\color{orange},
  numbers=left,
  numberstyle=\tiny,
  numbersep=5pt,
  tabsize=2,
  frame=single,
  breaklines=true,
  postbreak=\mbox{\textcolor{red}{$\hookrightarrow$}\space},
  showstringspaces=false,
}

\title{Quy hoạch động cơ bản: Prime 1}
\author{}
\date{}

\begin{document}

\maketitle

\section*{Đề bài}

Cho số nguyên dương $N$, hãy đếm xem trong đoạn từ $0$ tới $N$ có bao nhiêu số nguyên tố.

\textbf{Hướng dẫn}:
\begin{enumerate}
  \item \textbf{Bước 1:} Sàng số nguyên tố.
  \item \textbf{Bước 2:} Gọi $F[i]$ là số lượng các số nguyên tố từ $0$ tới $i$, xây dựng mảng $F[i]$ sau khi sàng.
\end{enumerate}

\section*{Input Format}

\begin{verbatim}
- Dòng 1 là số bộ test T
- T dòng tiếp theo mỗi dòng là 1 số nguyên không âm $N$
\end{verbatim}

\section*{Constraints}

\begin{verbatim}
- 1 <= T <= 10000
- 0 <= N <= $10^6$
\end{verbatim}

\section*{Output Format}

Đưa ra kết quả của mỗi test trên 1 dòng.
\\
\\
\begin{minipage}[t]{0.45\textwidth}
\begin{tcolorbox}[colback=green!10!white,colframe=green!50!black,title=Sample Input 0]
\begin{verbatim}
5
39
68
26
87
6
\end{verbatim}
\end{tcolorbox}
\end{minipage}
\hfill
\begin{minipage}[t]{0.45\textwidth}
\begin{tcolorbox}[colback=red!10!white,colframe=red!50!black,title=Sample Output 0]
\begin{verbatim}
12
19
9
23
3
\end{verbatim}
\end{tcolorbox}
\end{minipage}


\section*{C++ Code}

\begin{tcolorbox}[colback=blue!10!white,colframe=blue!50!black,title=\textbf{C++ Code}]
\begin{lstlisting}
#include <iostream>
#include <math.h>

using namespace std;

bool prime[1000001];
int F[1000001];

void sang();
int main() {
  int T;
  cin >> T;

  sang();
  while(T--) {
    int N;
    cin >> N;
    cout << F[N] << endl;
  }
  return 0;
}

void sang() {
  for(int i = 0; i < 1000001; i++) {
    prime[i] = true;
  }

  prime[0] = false;
  prime[1] = false;

  for(int i = 2; i <= sqrt(1000001); i++) {
    if(prime[i] == true) {
      for(int j = i * i; j < 1000001; j+= i) {
        prime[j] = false;
      }
    }
  }

  F[0] = 0;
  for(int i = 1; i < 1000001; i++) {
    if(prime[i] == true) {
      F[i] = F[i - 1] + 1;
    }
    else {
      F[i] = F[i - 1];
    }
  }
}
\end{lstlisting}
\end{tcolorbox}

\end{document}
